\documentclass{article}
\usepackage[utf8]{inputenc}
\usepackage{amsmath,amssymb,enumerate,xcolor,graphicx,amsthm,url,fdsymbol,tikz,enumitem}
\usepackage{algorithm, algpseudocode, float}
\usepackage{multirow, array}

 \graphicspath{ {./assets/} } 
\newcommand{\dist}[2]{#1\Leftrightarrow#2}
\newcommand{\image}[1]{\begin{figure}[H]
            \includegraphics[scale=.7]{#1}
            \centering
        \end{figure}}

\newcommand{\pay}[2]{\psi_{#1}\left(#2\right)}
\newcommand{\prob}[1]{\mathbb{P}\left( #1 \right)}
\newcommand{\pmf}[2]{p_{#1}\left( #2 \right)}
\newcommand{\pdf}[2]{f_{#1}\left( #2 \right)}
\newcommand{\cdf}[2]{F_{#1}\left( #2 \right)}
\newcommand{\modd}[1]{~\mathrm{mod}\,\left[ #1 \right]}
\newcommand{\expp}[1]{\mathbb{E}\left[ #1 \right]}
\newcommand{\varr}[1]{\mathbb{V}\left[ #1 \right]}
\newcommand{\covv}[1]{\mathbb{C}\left[ #1 \right]}

\title{CS 2850 -- Networks HW 4}
\author{jfw225 }
\date{October 2022}

\begin{document}

\maketitle

\begin{enumerate}
    \item \begin{enumerate}
            \item The largest strongly connected component of this graph is the set $\{5,6,7,8,9,10\}$.
            
            \item If we were to add a link from node 14 to node 2, the largest strongly connected component would become the set 
            $$\{2,3,4,5,6,7,8,9,10,11,12,14\}$$
            which is the largest possible set of strongly connected components from a single-link addition. Without loss of generality with respect to node 14 or 2, note that linking 14 to 1 is not the solution because 4 would be inaccessible. Similarly, linking 14 to 13 would do the same.
            
            \item From the original set that we mentioned in part (a), notice that each of those nodes depend on the link between 7 and 8 to be strongly connected. If we were to remove that link, there would be no strongly connected component with a larger size than one. Thus, deleting the link between 7 and 8 is the solution.

        \end{enumerate}

    \item After one iteration of the Basic PageRank Update Rule, we get the following table of values:
    \begin{center}\begin{tabular}{||c | c||} 
        \hline
        Node & Value \\ [0.5ex] 
        \hline\hline
        A & 4/17 \\
        \hline
        B & 3/17 \\
        \hline
        C & 3/17 \\
        \hline
        D & 2/17 \\
        \hline
        E & 4/17 \\
        \hline
        F & 1/17 \\
        \hline\hline
        \textbf{Sum} & 17/17 \\ [1ex] 
        \hline
    \end{tabular}\end{center}
    Although they do add to 1, the probabilities are not the same as the original values. Therefore, these numbers do not form an equilibrium.

    \pagebreak

    \item \begin{enumerate}
        \item The following table shows the value of each node in terms of $x$:
        \begin{center}\begin{tabular}{||c | c||} 
            \hline
            Node & Value in Terms of $x$ \\ [0.5ex] 
            \hline\hline
            A & $x$ \\
            \hline
            B & $x/2$ \\
            \hline
            C & $x/2$ \\
            \hline
            D & $x/4$ \\
            \hline
            E & $x/2$ \\
            \hline
            F & $x/4$ \\
            \hline
            G & $5x/8$ \\
            \hline
            H & $3x/8$ \\
            \hline
        \end{tabular}\end{center}

        \item We can solve for $x$ by summing these values and setting them equal to 1. We get the following equation:
        $$x + \frac{x}{2} + \frac{x}{2} + \frac{x}{4} + \frac{x}{2} + \frac{x}{4} + \frac{5x}{8} + \frac{3x}{8} = 1$$
        which yields $x=\frac{1}{4}$. Thus, our values become
        \begin{center}\begin{tabular}{||c | c||} 
            \hline
            Node & Value in Terms of $x$ \\ [0.5ex] 
            \hline\hline
            A & $\frac{1}{4}$ \\ [0.5ex]
            \hline
            B & $\frac{1}{8}$ \\ [0.5ex]
            \hline
            C & $\frac{1}{8}$ \\ [0.5ex]
            \hline
            D & $\frac{1}{16}$ \\ [0.5ex]
            \hline
            E & $\frac{1}{8}$ \\ [0.5ex]
            \hline
            F & $\frac{1}{16}$ \\ [0.5ex]
            \hline
            G & $\frac{5}{16}$ \\ [0.5ex]
            \hline
            H & $\frac{3}{16}$ \\ [0.5ex]
            \hline\hline
            \textbf{Sum} & 1 \\ [1ex] 
            \hline
        \end{tabular}\end{center}
    \end{enumerate}

    \item \begin{enumerate}
        \item After $k=2$ steps of the computation, we get the following table of values:
        \begin{center}\begin{tabular}{||c | c | c | c | c||} 
            \hline
            Node & Auth & Normalized Auth & Hub & Normalized Hub \\ [0.5ex] 
            \hline\hline
            A & 11 & 0.61 & 0 & 0.00 \\
            \hline
            B & 7 & 0.39 & 0 & 0.00 \\
            \hline
            C & 0 & 0.00 & 11 & 0.23 \\
            \hline
            D & 0 & 0.00 & 11 & 0.23 \\
            \hline
            E & 0 & 0.00 & 18 & 0.38 \\
            \hline 
            F & 0 & 0.00 & 7 & 0.15 \\
            \hline
        \end{tabular}\end{center}
    \end{enumerate}
   

\end{enumerate}

\end{document}
