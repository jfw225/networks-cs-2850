\documentclass{article}
\usepackage[utf8]{inputenc}
\usepackage{amsmath,amssymb,enumerate,xcolor,graphicx,amsthm,url,fdsymbol,tikz,enumitem}
\usepackage{algorithm, algpseudocode, float}
\usepackage{multirow, array}

 \graphicspath{ {./assets/} } 
\newcommand{\dist}[2]{#1\Leftrightarrow#2}
\newcommand{\image}[1]{\begin{figure}[H]
            \includegraphics[scale=.4]{#1}
            \centering
        \end{figure}}

\newcommand{\pay}[2]{\psi_{#1}\left(#2\right)}
\newcommand{\prob}[1]{\mathbb{P}\left( #1 \right)}
\newcommand{\pmf}[2]{p_{#1}\left( #2 \right)}
\newcommand{\pdf}[2]{f_{#1}\left( #2 \right)}
\newcommand{\cdf}[2]{F_{#1}\left( #2 \right)}
\newcommand{\modd}[1]{~\mathrm{mod}\,\left[ #1 \right]}
\newcommand{\expp}[1]{\mathbb{E}\left[ #1 \right]}
\newcommand{\varr}[1]{\mathbb{V}\left[ #1 \right]}
\newcommand{\covv}[1]{\mathbb{C}\left[ #1 \right]}

\title{CS 2850 -- Networks HW 7}
\author{jfw225}
\date{November 2022}

\begin{document}

\maketitle

\begin{enumerate}
    \item
    \begin{enumerate}
        \item Solving the equation yields equilibrium points of 0, 0.25, and 0.75.
        \item 0.25 is the only unstable equilibrium while the others are all stable.
        \item The product will become more popular over time because $z=0.3$ is between 0.25 and 0.75, and thus, it will grow in popularity over time.
        \item Our answer would not change it the value changed to $z=0.5$ because the function is still increasing between 0.25 and 0.75.
    \end{enumerate}
    \item
    \begin{enumerate}
        \item For $z=1/4$, we have
        \begin{align*}
            r(1/4)\,f(1/4) & =(1-1/4)\cdot\min(1,1) \\
            & =3/4.
        \end{align*}
        Since $3/4>7/16$, the fraction of interested population will be increasing. Therefore, $z=1/4$ is not an equilibrium.

        \item When $z<1/4$, we have
        \begin{align*}
            7/16=r(z)\,f(z) & =(1-z)\cdot\min(1,4z) \\
            7/16= & =(1-z)\cdot 4z \\
            7/16= & =4z-4z^2.
        \end{align*}
        Solving this equation yields $z=0,1/8$ as the equilibrium solutions.
        
        \item When $z>1/4$, we have
        \begin{align*}
            7/16=r(z)\,f(z) & =(1-z)\cdot\min(1,4z) \\
            7/16= & =(1-z)\cdot 1 \\
            7/16= & =1-z.
        \end{align*}
        Solving this equation yields $z=9/16$ as an equilibrium solution.
        
        \pagebreak
        \item We will observe that $z=1/4>1/8$ will grow to $z=9/16$ as shown in the following graph:
        
        \image{q2d.png}
    \end{enumerate}
    
    \item Let $n$ be the number of articles that it takes to receive $k$ views. Then it is theorized that $n=\frac{c}{k^{1.5}}$ for some constant $c$.
    \begin{enumerate}
        \item Let $n=200$ and $k=25$. Then we have
        $$n=\frac{c}{k^{1.5}} \implies 200=\frac{c}{25^{1.5}} \implies c=200\cdot25^{1.5}=25,000.$$
        \item Let $n=10$ and $k=\max(100,25)=100$. Then we have
        $$n=\frac{c}{k^{1.5}} \implies 10=\frac{c}{100^{1.5}} \implies c=10\cdot100^{1.5}=10,000.$$

        \pagebreak

        \item The question is equivalent to asking if there is some function $f$ whose parent is the power law such that $f(100)=40$ and $f(200)=10$. Using this information, we create the following two equations:
        \begin{align*}
            40 & =\frac{b}{100^a} \\
            10 & =\frac{b}{200^a} \\
            \implies 10 & =\frac{40\cdot100^a}{200^a} \\
            \frac{1}{4} & =\left(\frac{1}{2}\right)^a \\
            \implies a & =2 \\
            \implies b & =40\cdot100^2=400,000.
        \end{align*}
        Checking the answer, we have

        \image{q3c.png}
    \end{enumerate}

    \item
    \begin{enumerate}
        \item Observe that it is impossible for any graph that is produced by the \textsc{rich-get-richer} algorithm to contain a node who does not have exactly one outbound edge (other than node 1 because it is the oldest). Graph (a) could not have been produced from the algorithm because there are nodes that have more than one outbound edge. Graph (c) has a node other than node 1 with zero outbound edges. Thus, the only graphs that could have been produced by the algorithm are (b) and (d).
        
        \item A value of $p=1/2$ means that a node is equally as likely to connect to some node $i$ or copy the decision of node $i$. In graph (d), none of the nodes made the former choice--each node chose node $i$. Thus, it is more likely that (b) was produced by the algorithm.
    \end{enumerate}
\end{enumerate}

\end{document}
