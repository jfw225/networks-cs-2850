\documentclass{article}
\usepackage[utf8]{inputenc}
\usepackage{amsmath,amssymb,enumerate,xcolor,graphicx,amsthm,url,fdsymbol,tikz,enumitem}
\usepackage{algorithm, algpseudocode, float}
\usepackage{multirow, array}

 \graphicspath{ {./assets/} } 
\newcommand{\dist}[2]{#1\Leftrightarrow#2}
\newcommand{\image}[1]{\begin{figure}[H]
            \includegraphics[scale=.7]{#1}
            \centering
        \end{figure}}

\newcommand{\pay}[2]{\psi_{#1}\left(#2\right)}
\newcommand{\prob}[1]{\mathbb{P}\left( #1 \right)}
\newcommand{\pmf}[2]{p_{#1}\left( #2 \right)}
\newcommand{\pdf}[2]{f_{#1}\left( #2 \right)}
\newcommand{\cdf}[2]{F_{#1}\left( #2 \right)}
\newcommand{\modd}[1]{~\mathrm{mod}\,\left[ #1 \right]}
\newcommand{\expp}[1]{\mathbb{E}\left[ #1 \right]}
\newcommand{\varr}[1]{\mathbb{V}\left[ #1 \right]}
\newcommand{\covv}[1]{\mathbb{C}\left[ #1 \right]}

\title{CS 2850 -- Networks HW 3}
\author{jfw225 }
\date{September 2022}

\begin{document}

\maketitle

\begin{enumerate}
    \item Let $b_i$ be the bid made by the $i$-th bidder, and let $b_j$ be the bid made by the $j$-th bidder where $i,j\in\{2,3\}$ and $i\neq j$. Suppose that bidder $i$ bids at least as much as bidder $j$ (i.e. $b_i\geq b_j$). In order to win, we must bid $b_1>b_i$ for the item. Since this is a second-price sealed-bid auction, we will pay $b_i$ for the item if we win which yields a payoff of $v_1-b_i$. If instead we lose the auction, then our payoff is zero. From this, we consider two cases:
    \begin{itemize}
        \item \underline{\textbf{Case 1:} $\mathbf{b_i>v_1}$} If $b_i>v_1$ is the second highest bet and we won the auction, then it must be the case that we bid more than $v_1$. In this case, we are guaranteed to have a payoff of $v_1-b_i\leq0$. Notice that winning when $b_i>v_1$ is at best as valuable as not playing at all. Thus, we should never bid more than $v_1$ when $b_i>v_1$.
        \item \underline{\textbf{Case 2:} $\mathbf{b_i\leq v_1}$} In this case, we know that both $b_i$ and $b_j$ are at most $v_i$. We will still pay $b_i$ for the item, but we are guaranteed to have a payoff of at least zero. More specifically, our payoff is $v_1-b_i\geq0$. Notice that when $b_i\leq v_1$, a bid of $b_1=v_i$ is guaranteed to yield the exact same payoff as a bid of $b_1>v_i$.
    \end{itemize}

    Thus, bidding above $v_1=30$ never increases the payoff and only adds additional risk. Therefore, our friend is incorrect and we should never bid more than 30.

    \item Let $\pay{i}{R,v_i,b_i,b_j}$ be the payoff for bidder $i$ when placing a bid of $b_i$ for an object valued at $v_i$ while facing off in a two-buyer auction against some other bidder $j$ who places a bid of $b_j$. In addition, the auction is a second-price auction with a reserve price $R$. Let us further define this payoff function as 
    $$\pay{i}{R,v_i,b_i,b_j}=\begin{cases}
        0 & \text{if } b_i\leq b_j\text{ or } b_i\leq R, \\
        v_i-\max\{b_j,R\} & \text{if } b_i>b_i \text{ and } b_i>R.
    \end{cases}.$$
    \begin{enumerate}
        \item Let us refer to the opposing buyer as bidder $j$ and ourselves as bidder $i$. From the question, we know that $R=10$ and our value is $v_i=15$. Our goal is to bid such that our expected payoff is maximized, or rather, pick a value $b_i$ such that $\expp{\pay{i}{R,v_i,b_i,b_j}}_{b_j}$ is maximized, where the expected payoff is given by
        \begin{align*}
            \expp{\pay{i}{R,v_i,b_i,b_j}}_{b_j} & =\sum_{b\in\{5,10,15\}} \prob{b_j=b}\cdot\pay{i}{R=10,v_i=15,b_i,b_j=b}. \\
        \end{align*}
        Computing each value of $b_i$ gives us
        \begin{align*}
            \mathbf{b_i=5:} && \expp{\pay{i}{R,v_i,b_i=5,b_j}}_{b_j} & =\frac{1}{3}\cdot0+\frac{1}{3}\cdot0+\frac{1}{3}\cdot0 \\
            &&& =0, \\
            \mathbf{b_i=10:} && \expp{\pay{i}{R,v_i,b_i=10,b_j}}_{b_j} & =\frac{1}{3}(15-10)+\frac{1}{3}(15-10)+\frac{1}{3}\cdot0\\
            &&& =\frac{5}{3}, \\
            \mathbf{b_i=15:} && \expp{\pay{i}{R,v_i,b_i=15,b_j}}_{b_j} & =\frac{1}{3}(15-15)+\frac{1}{3}(15-10)+\frac{1}{3}\cdot0 \\
            &&& =\frac{5}{3}.
        \end{align*}
        Since choosing $b_i=15$ has a higher win probability than any other value and also yields an expected at least as good as the next choice, we should bid $b_i=15$.
        
        \item Let $m_0$ be the expected revenue earned from the second-price auction with a reserve price of $R=0$. We can compute $m_0$ by summing over all possible values of $\min\{b_i,b_j\}$ (note we take the min since this is a second-price auction) and multiplying by the probability of each value occurring. This gives us
        $$m_0=\sum_{b_i\in\{5,10,15\} }\sum_{b_j\in\{5,10,15\}}\frac{1}{9}\min\{b_i,b_j\}=\$7.78.$$

        \item If in this case $R=10$ and the expected revenue is $m_{10}$, then we only include values of $b_i,b_j$ if $b_i\geq10$ or $b_j\geq10$. This gives us
        $$m_{10}=m_0-\frac{1}{9}\min\{b_i=5,b_j=5\}=\$7.22.$$

        \item Let $m_p$ be the expected revenue earned for the object when the seller posts it for price $p$. We can compute $m_p$ by multiplying $p$ by the probability that at least one of the buyers want to buy the object at a price of at least $p$--let $E_p$ be this event. This gives us
        \begin{align*}
            m_p & =p\cdot\prob{E_p}, \\
            m_5 & =5\cdot\prob{E_5}=5\cdot\frac{9}{9}=\$5, \\
            m_{10} & =10\cdot\prob{E_{10}}=10\cdot\frac{8}{9}=\$8.89, \\
            m_{15} & =15\cdot\prob{E_{15}}=15\cdot\frac{5}{9}=\$8.33.
        \end{align*}
    \end{enumerate}

    \item 
    \begin{enumerate}
        \item The solution can be obtained by solving the following system of equations:
        \begin{align*}
            x+y & =100 \\
            10+\frac{x}{10} & =\frac{y}{20}+17 \\
        \end{align*}
        which yields a Nash equilibrium of $x=80$ and $y=20$.

        \item Let us refer to the new route (i.e. $A\to E\to B$) as Route III. Notice that $10+\frac{t}{10}<17+\frac{t}{20}$ for all $0\leq t\leq100$. Thus, it will always be faster to take Route III over Route II. In other words, Route II is strictly dominated by Route III. It follows that $y=0$ in the Nash equilibrium. We can determine the value of $x,z$ by solving the following system of equations:
        \begin{align*}
            x+z & =100 \\
            10+\frac{x}{10} & =\frac{z}{10}+10 \\
        \end{align*}
        which yields a Nash equilibrium of $x=50$, $y=0$, and $z=50$.
    \end{enumerate}

    \pagebreak

    \item
    \begin{enumerate}
        \item The following is a map of the game:
        \image{assets/q4a.png}

        Notice that taking Route 3 is strictly dominated if there exists a Nash equilibrium such that 
        \begin{align}
            4+\frac{x}{20} & \leq12, \label{eqn:eq1} \\ 
            2+\frac{y}{10} & \leq12. \label{eqn:eq2}
        \end{align} 
        If we let $z=0$ for now, then we can solve the following system of equations:
        \begin{align*}
            x+y & =200 \\    
            4+\frac{x}{20} & = 2+\frac{y}{10}
        \end{align*}
        which yields $x=120,y=80$. Since these values satisfy equations (\ref{eqn:eq1}) and (\ref{eqn:eq2}), we have a Nash equilibrium of $x=120,y=80,z=0$.

        \item
        \textbf{Project 1:} We can find a Nash equilibrium by solving the following system of equations:
        \begin{align*}
            x+y+z & =200 \\
            4+\frac{x}{20} & =2+\frac{y}{10} \\
            2+\frac{y}{10} & =5 
        \end{align*}
        which yields a Nash equilibrium of $x=20,y=30,z=150$ and a total travel time of $t_1=x\left(4+\frac{x}{20}\right) + y\left(2+\frac{y}{10}\right) + 5z =1000$ hours.

        \textbf{Project 2:} With the new road, travelers can simply go from $A\to C\to D\to B$ for a constant travel time of $4+0+2$ hours per traveler. This yields a Nash equilibrium of $x=0,y=0,z=0$ and a total travel time of $t_2=200*6=1200$ hours.

        Travelers could also start by taking the $A\to D$ path until $\frac{y}{10}=4$--at which point it becomes more efficient to take the $A\to C$ path. Thus, $y=40$ travelers will take the $A\to D$ path and 160 travelers will take the $A\to C$ path. From this point, all travelers have a choice to take the $C\to B$ path or the $D\to B$ path. Moreover, it is more efficient to take the $C\to B$ path until $\frac{x}{20}=2$--at which point it becomes more efficient to take the $D\to B$ path. Thus, $x=40$ travelers will take the $C\to B$ path and 160 travelers will take the $D\to B$ path. This yields a Nash equilibrium of $x=40,y=40,z=0$ and a total travel time of $t_2=160\cdot4+y\cdot\frac{y}{10}+160\cdot2+x\cdot\frac{x}{20}=200\cdot4+200\cdot2=1200$ hours.

        \textbf{Conclusion:} Project 1 should be chosen over Project 2 because it has a lower total travel time.
    \end{enumerate}

\end{enumerate}

\end{document}
