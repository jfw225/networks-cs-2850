\documentclass{article}
\usepackage[utf8]{inputenc}
\usepackage{amsmath,amssymb,enumerate,xcolor,graphicx,amsthm,url,fdsymbol,tikz,enumitem}
\usepackage{algorithm, algpseudocode, float}
\usepackage{multirow, array}

 \graphicspath{ {./assets/} } 
\newcommand{\dist}[2]{#1\Leftrightarrow#2}
\newcommand{\image}[1]{\begin{figure}[H]
            \includegraphics[scale=.4]{#1}
            \centering
        \end{figure}}

\newcommand{\pay}[2]{\psi_{#1}\left(#2\right)}
\newcommand{\prob}[1]{\mathbb{P}\left( #1 \right)}
\newcommand{\pmf}[2]{p_{#1}\left( #2 \right)}
\newcommand{\pdf}[2]{f_{#1}\left( #2 \right)}
\newcommand{\cdf}[2]{F_{#1}\left( #2 \right)}
\newcommand{\modd}[1]{~\mathrm{mod}\,\left[ #1 \right]}
\newcommand{\expp}[1]{\mathbb{E}\left[ #1 \right]}
\newcommand{\varr}[1]{\mathbb{V}\left[ #1 \right]}
\newcommand{\covv}[1]{\mathbb{C}\left[ #1 \right]}

\title{CS 2850 -- Networks HW 8}
\author{jfw225}
\date{November 2022}

\begin{document}

\maketitle

\begin{enumerate}
   \item \begin{enumerate}
        \item One iteration after the new behavior is introduced, we will see node 3 adopt $A$ because it only has three neighbors. On the next iteration, nodes 9 and 4 will also adopt $A$ because $1/3,2/5>0.3$ of their neighbors respectively have adopted $A$. This will continue until every node on the left hand side of the graph has adopted $A$. At this point, adoption will cease because $1/4<0.3$ of the neighbors of node 6 have adopted $A$.
        
        \item The sets are constituted by the two sides of the graph:
        $$S=\{1,3,4,5,9,10\}\text{ and }T=\{2,6,7,8,11\}.$$

        \item Under the current adoption rule, this is not possible. The only way to get from node 6 to node 5 (or vice versa) is to have an adoption rule such that a node will adopt $A$ if at least $1/4$ of its neighbors have adopted $A$. Thus, changing the starting node will not impact the ability for $A$ to be adopted between nodes $5,6$.
   \end{enumerate}

   \item \begin{enumerate}
        \item We would observe this pattern with a value of $q=1/4$. In the first iteration, we would see nodes 2 and 4 switch to $A$, and then in the next iteration, we would see nodes $3,5,7,8$ switch to $A$. Finally, in the last iteration, we would see node 6 switch to $A$.
        \item Firstly, we see that node 2 has 3 neighbors and node 4 has 4 neighbors. Thus, choosing $q=1/3$ would lead to node 2 adopting $A$ in the first iteration, and node 4 not adopting $A$ until node $3$ has adopted $A$. The rest of the pattern follows from running the rest of the iterations.
        \item Suppose for contradiction that $q$ exists and node 2 adopts $A$ on the first iteration. For the desired pattern to be observed, it must be the case that $q<1/2$ for node 5 to adopt $A$ on the second iteration, and $q>2/5$ for node 3 not to adopt $A$. However, if $q>2/5$, then node 2 will not adopt $A$ on the first iteration--therein lies the contradiction. Therefore, there is no value of $q$ that exists which yields the desired pattern. 
   \end{enumerate}

   \item From the question, it must be the case that $u$ should choose $A$ if $x>q$ and should choose $B$ if $x<q$ (we ignore the case that $x=q$). Then we can answer this question by treating $x$ as the probability that $u$ chooses $A$ and solving for $x$ using the payoff matrix:
   \begin{align*}
        5x-2(1-x) & >1x-3(1-x) \\
        \implies x & >\frac{1}{5}.
   \end{align*}
   Thus, $u$ should choose $A$ if $x>\frac{1}{5}$ and should choose $B$ if $x<\frac{1}{5}$. Therefore, $q=\frac{1}{5}$.

   \item \begin{enumerate}
        \item We can find a value for $q$ by looking at the ratio of friends in each region. More specifically, in order for the new feature to not spread from $X$, to $Y$, it must be the case that $q$ is greater than the number of friends that users in $Y$ have in $X$ divided by the number of friends that users in $X$ have in $X$, or rather, $q>\frac{1}{6}$. Moreover, for the new feature to spread from $Y$ to $X$, it must be the case that $q$ is at most the number of friends that users in $X$ have in $Y$ divided by the total number of friends that users in $X$ have in both regions, or rather, $q\leq\frac{4}{16}=\frac{1}{4}$. Thus, the desired value of $q$ exists and is $\frac{1}{6}<q<\frac{1}{4}$.
        
        \item The desired pattern could be observed by taking the inverse of the situation that we described in part (a). This yields a value of $\frac{1}{4}<q\leq\frac{1}{6}$. However, this is not possible.
   \end{enumerate}
\end{enumerate}

\end{document}
