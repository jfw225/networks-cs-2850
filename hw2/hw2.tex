\documentclass{article}
\usepackage[utf8]{inputenc}
\usepackage{amsmath,amssymb,enumerate,xcolor,graphicx,amsthm,url,fdsymbol,tikz,enumitem}
\usepackage{algorithm, algpseudocode, float}
\usepackage{multirow, array}

 \graphicspath{ {./assets/} } 
\newcommand{\dist}[2]{#1\Leftrightarrow#2}
\newcommand{\image}[1]{\begin{figure}[H]
            \includegraphics[scale=.4]{#1}
            \centering
        \end{figure}}

\newcommand{\pay}[3]{\psi\left(#1,#2,#3\right)}
\newcommand{\prob}[1]{\mathbb{P}\left( #1 \right)}
\newcommand{\pmf}[2]{p_{#1}\left( #2 \right)}
\newcommand{\pdf}[2]{f_{#1}\left( #2 \right)}
\newcommand{\cdf}[2]{F_{#1}\left( #2 \right)}
\newcommand{\modd}[1]{~\mathrm{mod}\,\left[ #1 \right]}
\newcommand{\expp}[1]{\mathbb{E}\left[ #1 \right]}
\newcommand{\varr}[1]{\mathbb{V}\left[ #1 \right]}
\newcommand{\covv}[1]{\mathbb{C}\left[ #1 \right]}

\title{CS 2850 -- Networks}
\author{jfw225 }
\date{August 2022}

\begin{document}

\maketitle

\begin{enumerate}
    \item 
    Let $\pay{\mathcal{P}}{\mathcal{R}}{\mathcal{C}}$ be the payoff for player 
    $\mathcal{P}\in\left\{A,B\right\}$ for row $\mathcal{R}\in\left\{U,D\right\}$ 
    and column $\mathcal{C}\in\left\{L,R\right\}$. Additionally, let $\expp{\pay{\mathcal{P}}{\mathcal{R}}{\mathcal{C}}}$ be the expected payoff for player $\mathcal{P}$ for row $\mathcal{R}$ and column $\mathcal{C}$.
    
    \begin{enumerate}
        \item The dominant strategy for Player A is always to choose row $D$ 
        because they will always have a better payoff than if they picked row 
        $U$, regardless of the column chosen by player B. More rigidly, $\pay{A}{D}{\mathcal{C}}>\pay{A}{U}{\mathcal{C}}$ for all $\mathcal{C}\in\left\{L,R\right\}$.

        Likewise, the dominant strategy for Player B is always to choose row $R$ because $\pay{B}{\mathcal{R}}{R}>\pay{B}{\mathcal{R}}{L}$ for all $\mathcal{R}\in\left\{U,D\right\}$.

        Therefore, the Nash equilibrium is $(D,R)$.

        \item Unlike the in part (a), Player A does not have a dominant strategy. That is, Player A cannot guarantee a better payoff by simply always picking some row. 
        
        However, Player B does have a dominant strategy, which is to always pick column $R$. This is because $\pay{B}{\mathcal{R}}{R}>\pay{B}{\mathcal{R}}{L}$ for all $\mathcal{R}\in\left\{U,D\right\}$.

        Given that Player B will always choose column $R$, Player A is better off choosing row $U$ because $\pay{A}{U}{R}>\pay{A}{D}{R}$. Since there is no change in strategy that will result in a better payoff for Player A, $(U,R)$ is the Nash equilibrium.

        \item First, observe that there is no pure strategy that is a part of the Nash equilibrium for this game. That is, there is no strategy that will result in a better payoff for both players. Thus, we must consider mixed-strategies. Let $p$ be the probability that Player A chooses row $U$, and $q$ be the probability that Player B chooses column $L$. Then we can write the expected payoffs for each player in terms of $p,q$:
        \begin{align*}
            \textbf{\underline{Player A:}} && \expp{\pay{A}{U}{\mathcal{C}}} & =q\cdot\pay{A}{U}{L} + (1-q)\cdot\pay{A}{U}{R} \\
            &&& =q+(1-q)\cdot0=q; \\
            && \expp{\pay{A}{D}{\mathcal{C}}} & =q\cdot\pay{A}{D}{L} + (1-q)\cdot\pay{A}{D}{R} \\
            &&& =q\cdot0+(1-q)\cdot1=1-q; \\
            \textbf{\underline{Player B:}} && \expp{\pay{B}{\mathcal{R}}{L}} & =p\cdot\pay{B}{U}{L} + (1-p)\cdot\pay{B}{D}{L} \\\
            &&& =p\cdot1+(1-p)\cdot2=1-p; \\
            && \expp{\pay{B}{\mathcal{R}}{R}} & =p\cdot\pay{B}{U}{R} + (1-p)\cdot\pay{B}{D}{R} \\
            &&& =p\cdot2+(1-p)\cdot1=p+1.
        \end{align*}

        From section 6.7 of the textbook, we know that $\expp{\pay{A}{U}{\mathcal{C}}}=\expp{\pay{A}{D}{\mathcal{C}}}$ and $\expp{\pay{B}{\mathcal{R}}{L}}=\expp{\pay{B}{\mathcal{R}}{R}}$. If this were not the case, we would have a contradiction because we established that there are no pure strategies. Thus, we can solve for $p,q$:
        \begin{align*}
            \expp{\pay{A}{U}{\mathcal{C}}} & =\expp{\pay{A}{D}{\mathcal{C}}} \\
            q & =1-q \\
            \Rightarrow q & =\frac{1}{2}; \\
            \expp{\pay{B}{\mathcal{R}}{L}} & =\expp{\pay{B}{\mathcal{R}}{R}} \\
            1-p & =p+1 \\
            \Rightarrow p & =\frac{1}{2}.
        \end{align*}

        Therefore, the mixed-strategy Nash equilibrium is $(p=\frac{1}{2},q=\frac{1}{2})$.
    \end{enumerate}

    \item Without loss of generality with respect to the starting player, suppose that Player A goes first and chooses row $U$ each time. Player $B$ will always choose column $L$ because $\pay{B}{U}{L}>\pay{B}{U}{R}$. Notice that from the state $(U,L)$, Player A switching strategies to $D$ will result in a worse payoff. Likewise, Player B switching strategies to $R$ will result in a worse payoff. Thus, $(U,L)$ is a Nash equilibrium that is constituted by pure strategies.
    
    On the other hand, suppose that Player A chooses row $D$ each time. Player $B$ will always choose column $R$ because $\pay{B}{D}{R}>\pay{B}{D}{L}$. Once again, Player A switching strategies to $U$ will result in a worse payoff. Likewise, Player B switching strategies to $L$ will result in a worse payoff. Thus, $(D,R)$ is a Nash equilibrium that is constituted by pure strategies.

    Although there exist equilibria that are constituted by pure strategies, each Player's best response is dependent on the other Player's strategy. Thus, neither player has a strategy that is uniquely the best response to the other player's strategy. Thus, neither play has a dominant strategy and a mixed-strategy equilibrium exists. We can solve for the mixed-strategy Nash equilibrium in the same way as in part (c) of the previous question (let $p$ and $q$ be defined as before):
    \begin{align*}
        \textbf{\underline{Player A:}} && \expp{\pay{A}{U}{\mathcal{C}}} & =\expp{\pay{A}{D}{\mathcal{C}}} \\
        && q\cdot\pay{A}{U}{L} + (1-q)\cdot\pay{A}{U}{R} & =q\cdot\pay{A}{D}{L} + (1-q)\cdot\pay{A}{D}{R} \\
        && 5q + 3(1-q) & =4q + 7(1-q) \\
        && 2q+3 & =7-3q \\
        && q & =\frac{4}{5}; \\
        \textbf{\underline{Player B:}} && \expp{\pay{B}{\mathcal{R}}{L}} & =\expp{\pay{B}{\mathcal{R}}{R}} \\
        && p\cdot\pay{B}{U}{L} + (1-p)\cdot\pay{B}{D}{L} & =p\cdot\pay{B}{U}{R} + (1-p)\cdot\pay{B}{D}{R} \\
        && 1p+1(1-p) & =0p+2(1-p) \\
        && p & =\frac{1}{2}.
    \end{align*}

    Therefore, there is also a mixed-strategy Nash equilibrium of 
    $$(p=\frac{1}{2},q=\frac{4}{5}).$$

    \item Let $\pay{\mathcal{P}}{\mathcal{R}}{\mathcal{C}}$ be the payoff for player $\mathcal{P}\in\left\{A,B\right\}$ for row $\mathcal{R}\in\left\{X,Y,Z\right\}$ and column $\mathcal{C}\in\left\{R,S,T,U\right\}$.
    \begin{enumerate}
        \item Suppose that Player A starts and chooses row $X$. Then Player B will choose column $R$ because it has the highest payoff. Now suppose that instead, Player A starts and chooses row $Y$. Then Player B will choose column $T$ because it has the highest payoff. Thus, Player B does not have a dominant strategy. 
        
        On the other hand, suppose that Player B starts and chooses column $S$. Then Player A will choose row $X$ because it has the highest payoff. Now suppose that instead, Player B starts and chooses column $T$. Then Player A will choose row $Y$ because it has the highest payoff. Thus, Player A does not have a dominant strategy.
        
        Therefore, neither player has a dominant strategy.

        \item Notice that $\pay{A}{Y}{\mathcal{C}}>\pay{A}{Z}{\mathcal{C}}$ for all $\mathcal{C}\in\left\{R,S,T,U\right\}$. More clearly, Player A will always choose row $Y$ over row $Z$ regardless of the column because it has a higher payoff. Thus, strategy $Z$ is strictly dominated by strategy $Y$. This is the only strictly dominated strategy for Player A.
        
        For Player B, we observe that several strategies that fit this criteria: 
        \begin{align*}
            \pay{B}{\mathcal{R}}{T} & >\pay{B}{\mathcal{R}}{S} \\
            \pay{B}{\mathcal{R}}{T} & >\pay{B}{\mathcal{R}}{U} \\
            \pay{B}{\mathcal{R}}{R} & >\pay{B}{\mathcal{R}}{U} \\
        \end{align*}
        for all $\mathcal{R}\in\left\{X,Y,Z\right\}$. Therefore, strategies $S$ and $U$ are strictly dominated by strategy $T$, and strategy $U$ is also strictly dominated by strategy $R$. Thus, there are three strictly dominated strategies for Player B.

        \item Using my answer in (b) to help, we can see that whenever Player A chooses row $X$, Player $B$ will choose column $R$. Since neither player is better off choosing another strategy from this state, $(X,R)$ is a Nash equilibrium that is constituted by pure strategies. Similarly, whenever Player A chooses row $Y$, Player $B$ will choose column $T$. Since neither player is better off choosing another strategy from this state, $(Y,T)$ is also a Nash equilibrium that is constituted by pure strategies. Notice that strategy $Z$ can never be a part of a Nash equilibrium because it is strictly dominated by strategy $Y$.
    \end{enumerate}

    \item 
    \begin{enumerate}
        \item The payoff matrix is as follows:

        \begin{table}[H]
            \centering
            \setlength{\extrarowheight}{2pt}
            \begin{tabular}{cc|c|c|}
              & \multicolumn{1}{c}{} & \multicolumn{2}{c}{Defender}\\
              & \multicolumn{1}{c}{} & \multicolumn{1}{c}{$A$}  & \multicolumn{1}{c}{$B$} \\\cline{3-4}
              \multirow{2}*{Attacker}  & $A$ & $(l,w)$ & $(w,l)$ \\\cline{3-4}
              & $B$ & $(w,l)$ & $(l,w)$ \\\cline{3-4}
            \end{tabular}
        \end{table}
    
        \item Since there is no state in which both parties can win, the losing player is always better off picking the strategy that they did not choose. Thus, there is no pure strategy equilibrium for this game.
        
        \item Since there is no pure strategy equilibrium, we can use the same method as in (2) to find the mixed-strategy equilibrium. Let $p$ be the probability that the attacker chooses strategy $A$, and let $q$ be the probability that the defender chooses strategy $A$:
        \begin{align*}
            \textbf{\underline{Player X:}} && l\cdot q + w\cdot (1-q) & =w\cdot q + l\cdot (1-q) \\
            && 2lq-2wq & =l-w \\
            && q & =\frac{(l-w)}{2(l-w)}=\frac{1}{2}; \\
            \textbf{\underline{Player Y:}} && w\cdot p + l\cdot (1-p) & =l\cdot p + w\cdot (1-p) \\
            && 2w\cdot p - w & =2l\cdot p+l \\
            && 2wp-2lp & =w-l \\
            && p & =\frac{(w-l)}{2(w-l)}=\frac{1}{2}.
        \end{align*}
        Solving these equations yields $p=q=\frac{1}{2}$, which is the mixed-strategy equilibrium.

        \item When solving above, we can see that the parts of the equations that contain $l$ and $w$ cancel out. Thus, the actual values of $l$ and $w$ do not impact the probabilities.

\end{enumerate}

    \item 
    \begin{enumerate}
        \item Below is the graph described by the given rule where blue edges indicate two nodes are allies and red edges indicate that two nodes are enemies: 
        \image{assets/q5a.png}
        If there exists a set of three nodes such that there is only one edge between enemies and two edges between allies, then the graph is not structurally balanced. This occurs at least once in the graph above, namely, for $A,F,H$:

        \image{assets/q5a2.png}

        Therefore, this graph is not structurally balanced.
        \pagebreak

        \item The following graph shows this relationship where every edge between two nodes indicates an alliance:
        
        \image{assets/q5b.png}

        According to the Balance Theorem from the textbook, this graph is structurally balanced because we can split the nodes into two groups $X=\{A,B,C,D,E,F,G,H\}$ and $Y=\{I,J,K,L\}$ such that "each pair of people in $X$ likes each other, each pair of people in $Y$ likes each other, and everyone in $X$ is the enemy of everyone in $Y$." Thus, this graph is structurally balanced.

        \item Under the given constraint, it would be the case that $J$ is an enemy of $K$, $K$ is an enemy of $L$, and $L$ is an enemy of $J$. Thus, there is a set of three nodes that are all mutual enemies. Therefore, this graph is not structurally balanced.
        
        \pagebreak
        \item The following graph shows this relationship where every edge between two nodes indicates an alliance:
        \image{assets/q5d.png}

        According to the Balance Theorem from the textbook, this graph is structurally balanced because we can split the nodes into two groups $X=\{A,B,C,D,E,F,G,H,I,J,K\}$ and $Y=\{L\}$ such that "each pair of people in $X$ likes each other, each pair of people in $Y$ likes each other, and everyone in $X$ is the enemy of everyone in $Y$." Thus, this graph is structurally balanced.
    \end{enumerate}
\end{enumerate}

\end{document}
