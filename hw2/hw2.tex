\documentclass{article}
\usepackage[utf8]{inputenc}
\usepackage{amsmath,amssymb,enumerate,xcolor,graphicx,amsthm,url,fdsymbol,tikz,enumitem}
\usepackage{algorithm, algpseudocode, float}

 \graphicspath{ {./assets/} } 
\newcommand{\dist}[2]{#1\Leftrightarrow#2}
\newcommand{\image}[1]{\begin{figure}[H]
            \includegraphics[scale=.65]{#1}
            \centering
        \end{figure}}

\newcommand{\pay}[3]{\psi\left(#1,#2,#3\right)}
\newcommand{\prob}[1]{\mathbb{P}\left( #1 \right)}
\newcommand{\pmf}[2]{p_{#1}\left( #2 \right)}
\newcommand{\pdf}[2]{f_{#1}\left( #2 \right)}
\newcommand{\cdf}[2]{F_{#1}\left( #2 \right)}
\newcommand{\modd}[1]{~\mathrm{mod}\,\left[ #1 \right]}
\newcommand{\expp}[1]{\mathbb{E}\left[ #1 \right]}
\newcommand{\varr}[1]{\mathbb{V}\left[ #1 \right]}
\newcommand{\covv}[1]{\mathbb{C}\left[ #1 \right]}

\title{CS 2850 -- Networks}
\author{jfw225 }
\date{August 2022}

\begin{document}

\maketitle

\begin{enumerate}
    \item 
    Let $\pay{\mathcal{P}}{\mathcal{R}}{\mathcal{C}}$ be the payoff for player 
    $\mathcal{P}\in\left\{A,B\right\}$ for row $\mathcal{R}\in\left\{U,D\right\}$ 
    and column $\mathcal{C}\in\left\{L,R\right\}$. Additionally, let $\expp{\pay{\mathcal{P}}{\mathcal{R}}{\mathcal{C}}}$ be the expected payoff for player $\mathcal{P}$ for row $\mathcal{R}$ and column $\mathcal{C}$.
    
    \begin{enumerate}
        \item The dominant strategy for Player A is always to choose row $D$ 
        because they will always have a better payoff than if they picked row 
        $U$, regardless of the column chosen by player B. More rigidly, $\pay{A}{D}{\mathcal{C}}>\pay{A}{U}{\mathcal{C}}$ for all $\mathcal{C}\in\left\{L,R\right\}$.

        Likewise, the dominant strategy for Player B is always to choose row $R$ because $\pay{B}{\mathcal{R}}{R}>\pay{B}{\mathcal{R}}{L}$ for all $\mathcal{R}\in\left\{U,D\right\}$.

        Therefore, the Nash equilibrium is $(D,R)$.

        \item Unlike the in part (a), Player A does not have a dominant strategy. That is, Player A cannot guarantee a better payoff by simply always picking some row. 
        
        However, Player B does have a dominant strategy, which is to always pick column $R$. This is because $\pay{B}{\mathcal{R}}{R}>\pay{B}{\mathcal{R}}{L}$ for all $\mathcal{R}\in\left\{U,D\right\}$.

        Given that Player B will always choose column $R$, Player A is better off choosing row $U$ because $\pay{A}{U}{R}>\pay{A}{D}{R}$. Since there is no change in strategy that will result in a better payoff for Player A, $(U,R)$ is the Nash equilibrium.

        \item First, observe that there is no pure strategy that is a part of the Nash equilibrium for this game. That is, there is no strategy that will result in a better payoff for both players. Thus, we must consider mixed-strategies. Let $p$ be the probability that Player A chooses row $U$, and $q$ be the probability that Player B chooses column $L$. Then we can write the expected payoffs for each player in terms of $p,q$:
        \begin{align*}
            \textbf{\underline{Player A:}} && \expp{\pay{A}{U}{\mathcal{C}}} & =q\cdot\pay{A}{U}{L} + (1-q)\cdot\pay{A}{U}{R} \\
            &&& =q+(1-q)\cdot0=q; \\
            && \expp{\pay{A}{D}{\mathcal{C}}} & =q\cdot\pay{A}{D}{L} + (1-q)\cdot\pay{A}{D}{R} \\
            &&& =q\cdot0+(1-q)\cdot1=1-q; \\
            \textbf{\underline{Player B:}} && \expp{\pay{B}{\mathcal{R}}{L}} & =p\cdot\pay{B}{U}{L} + (1-p)\cdot\pay{B}{D}{L} \\\
            &&& =p\cdot1+(1-p)\cdot2=1-p; \\
            && \expp{\pay{B}{\mathcal{R}}{R}} & =p\cdot\pay{B}{U}{R} + (1-p)\cdot\pay{B}{D}{R} \\
            &&& =p\cdot2+(1-p)\cdot1=p+1.
        \end{align*}

        From section 6.7 of the textbook, we know that $\expp{\pay{A}{U}{\mathcal{C}}}=\expp{\pay{A}{D}{\mathcal{C}}}$ and $\expp{\pay{B}{\mathcal{R}}{L}}=\expp{\pay{B}{\mathcal{R}}{R}}$. If this were not the case, we would have a contradiction because we established that there are no pure strategies. Thus, we can solve for $p,q$:
        \begin{align*}
            \expp{\pay{A}{U}{\mathcal{C}}} & =\expp{\pay{A}{D}{\mathcal{C}}} \\
            q & =1-q \\
            \Rightarrow q & =\frac{1}{2}; \\
            \expp{\pay{B}{\mathcal{R}}{L}} & =\expp{\pay{B}{\mathcal{R}}{R}} \\
            1-p & =p+1 \\
            \Rightarrow p & =\frac{1}{2}.
        \end{align*}

        Therefore, the mixed-strategy Nash equilibrium is $(p=\frac{1}{2},q=\frac{1}{2})$.
    \end{enumerate}

    \item Without loss of generality with respect to the starting player, suppose that Player A goes first and chooses row $U$ each time. Player $B$ will always choose column $L$ because $\pay{B}{U}{L}>\pay{B}{U}{R}$. Notice that from the state $(U,L)$, Player A switching strategies to $D$ will result in a worse payoff. Likewise, Player B switching strategies to $R$ will result in a worse payoff. Thus, $(U,L)$ is a Nash equilibrium that is constituted by pure strategies.
    
    On the other hand, suppose that Player A chooses row $D$ each time. Player $B$ will always choose column $R$ because $\pay{B}{D}{R}>\pay{B}{D}{L}$. Once again, Player A switching strategies to $U$ will result in a worse payoff. Likewise, Player B switching strategies to $L$ will result in a worse payoff. Thus, $(D,R)$ is a Nash equilibrium that is constituted by pure strategies.

    Although there exist equilibria that are constituted by pure strategies, each Player's best response is dependent on the other Player's strategy. Thus, neither player has a strategy that is uniquely the best response to the other player's strategy. Thus, neither play has a dominant strategy and a mixed-strategy equilibrium exists. We can solve for the mixed-strategy Nash equilibrium in the same way as in part (c) of the previous question (let $p$ and $q$ be defined as before):
    \begin{align*}
        \textbf{\underline{Player A:}} && \expp{\pay{A}{U}{\mathcal{C}}} & =\expp{\pay{A}{D}{\mathcal{C}}} \\
        && q\cdot\pay{A}{U}{L} + (1-q)\cdot\pay{A}{U}{R} & =q\cdot\pay{A}{D}{L} + (1-q)\cdot\pay{A}{D}{R} \\
        && 5q + 3(1-q) & =4q + 7(1-q) \\
        && 2q+3 & =7-3q \\
        && q & =\frac{4}{5}; \\
        \textbf{\underline{Player B:}} && \expp{\pay{B}{\mathcal{R}}{L}} & =\expp{\pay{B}{\mathcal{R}}{R}} \\
        && p\cdot\pay{B}{U}{L} + (1-p)\cdot\pay{B}{D}{L} & =p\cdot\pay{B}{U}{R} + (1-p)\cdot\pay{B}{D}{R} \\
        && 1p+1(1-p) & =0p+2(1-p) \\
        && p & =\frac{1}{2}.
    \end{align*}

    Therefore, there is also a mixed-strategy Nash equilibrium of 
    $$(p=\frac{1}{2},q=\frac{4}{5}).$$
\end{enumerate}

\end{document}
